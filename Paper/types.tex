\subsection{Types}\label{sec:Types}
Like \C/\Cpp, \llvm{} instructions need to know the types of the data
on which they operate.  The instructions are specific to particular
types.  Similarly, we need to specify the types of parameters, local
variables and the return type of a routine.
Therefore, %in order to compile code with \llvm{}, 
we need to know the types of variables and parameters.  This is quite different
from \R's dynamic, run-time type information.

The \llvm{} API and \Rllvm{} provide numerous facilities for working with types.
Firstly, we have an \R{} variable for each of the basic or primitive
types.  These are \Rvar{Int32Type} corresponding to a $4$-byte
integer, \Rvar{DoubleType} for a $8$-byte real valued number,
\Rvar{FloatType} for a $4$-byte real valued number.  Thes correspond
to \Ctype{int}, \Ctype{double} and \Ctype{float} types in \C. There
are other variants of integers such as \Rvar{Int8Type},
\Rvar{Int1Type} and \Rvar{Int64Type}.  \Rvar{Int1Type} is a single
bit. \Rvar{Int8Type} is a one byte integer which also
corresponds to a single character type.
The \Rvar{StringType} variable represents
a string, which is a sequence of characters.

We can create aggregate types, or structures, using
the \Rfunc{structType} function.
We specify the types of each of the elements in the order
they are to be arranged. 
\begin{RCode}
studentType = structType(list(id = StringType,
                              age = Int8Type,
                              gpa = DoubleType,
                             ),  "Student")  
\end{RCode}
Here we have both names for the elements and also for the aggregate
type (\Ctype{Student}).  \llvm{} doesn't actually care about
these. However, in \R{}, we typically do need this information and so
we store it in the type object returned by \Rfunc{structType}..


We often need pointer types.  There is an \R{} variable for a pointer
type for each of the primitive types. \Rvar{DoublePtrType} is a
pointer to a \Ctype{double}, \Rvar{Int32PtrType} for 32 bit integers,
and so on.  We can create a type that is a pointer to  any other
type using \Rfunc{pointerType}.
For example, to define a pointer to our \Ctype{Student}  type above,
we use
\begin{RCode}
studentPtrType = pointerType( studentType )  
\end{RCode}
To create a pointer to a pointer to a \Ctype{Student} type,
we can use either of 
\begin{RCode}
studentPtrType = pointerType( studentPtrType )  
studentPtPtrType = pointerType( pointerType( studentType )  )
\end{RCode}


We can define array types  by specifying the
type of each element and the number of elements.
For example, to create an array of 10 \Ctype{double} values,
we can use
\begin{RCode}
arrayType(DoubleType, 10)  
\end{RCode}
or for an array 100 \Ctype{Student} elements
\begin{RCode}
arrayType(studentType, 100)    
\end{RCode}


Every type object has an identifier which tells us the nature of the
type.  We can obtain this with \Rfunc{getTypeID} and this 
returns a value such as 
\begin{ROutput}
IntegerTyID 
         10   
\end{ROutput}
The name indicates that this is an integer type.
The value $10$  corresponds to a \C-level enumeration for the type kinds.


There are numerous functions in \Rllvm{} to query 
aspects of an existing type.
\Rfunc{getElementType} returns the target type of a pointer.
Similarly, \Rfunc{getArrayType}
\Rfunc{getElementTypes} returns the types of fields in a structure type.

Predicate functions such as \Rfunc{isVoidType}, \Rfunc{isPointerType}
and \Rfunc{isStringType} query whether the specified type corresponds
to a particular type.  \Rfunc{sameType} is usef for comparing if two
types are really the same. This is useful because sometimes we compare
types represented as \R{} objects and pointers to \llvm{} types.
\Rfunc{sameType} hides these details and allows us to compare two
types easily.


Essentialy, ever element in \llvm{} corresponding to an instruction or
a variable or parameter has a type. We can query it with
\Rfunc{getType}.



\paragraph{\llvm{} types corresponding to \R{} data types}
When we compile \R{} code, we often want to be able to operate on \R's
own data types, e.g. \Rclass{logical}, \Rclass{integer},
\Rclass{numeric}, \Rclass{complex}, \Rclass{character} and \Rclass{list}.
(There are others.)  These are different varieties of \R's
\Ctype{SEXP} (Symbolic Expression) type.  \Rllvm{} defines types
corresponding to these in the variables \Rvar{LGLSXPType},
\Rvar{INTSXPType}, \Rvar{REALSXPType}, \Rvar{CPLXSXPType} and
\Rvar{VECSXPType}, corresponding to the names of the enumerated
constants describing each type in \R's \C{} code.  There is also the
generic \Rvar{SEXPType}, meaning any \R{} type.  Currently, all of
these types are the same and are generic \R{} types.  This means we
cannot distinguish between them at the \llvm{} level. This is
important so that the code we generate doesn't differentiate between
them.  We encourage using the more specific types when writing \R{}
code so that it is clear what is intended, i.e.  an integer via
\Rvar{INTSXPType} rather than a \Rvar{REALSXPType}.  In the near
future, we hope to make these separate types at the \llvm-level so
that we can tell them apart.

% 
% Perhaps a table of the basic types.


%Basic types
%array types
%pointers
%structs

%SEXP types.
